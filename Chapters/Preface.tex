\chapter{Bevezetés}
\setcounter{page}{1}
\pagenumbering{arabic} 
Az élet szinte minden területén találkozhatunk ütemezési problémákkal, például a logisztika területén, ha csak arra gondolunk, hogy egy futárszolgálatnak meg kell tervezni az optimális kiszállítási időpontokat, az útvonalat annak érdekében, hogy időben kézhez kaphassunk az általunk rendelt termékeket, de vehetjük példának akár a tömegközlekedést is, ahol a menetrendek megtervezése szintén ütemezési feladat.
Számtalan példát fel lehetne még sorolni, azonban az belátható, hogy bármennyire is eltérhetnek ezen példák valamely aspektus szerint, ezek lényege minden esetben megegyezik: a cél az elvégzendő feladatokat szétosztása az elérhető erőforrások között, valamint ezen feladatok egy idő intervallumhoz rendelése oly módon, hogy a keletkező ütemterv valamilyen célkitűzés szempontjából a legkedvezőbb legyen, valamint a probléma definiálása során meghatározott korlátozásoknak megfeleljen.
Az ütemezési feladatok egyik leggyakoribb megjelenési formája az iparban a gyártórendszerek ütemezése, mely során fellépő problémák megoldására számos megoldó módszer fellelhető a szakirodalomban.
Ezen problémák csoportosításáról, néhány megoldó módszer ismertetéséről, valamint az ütemtervek vizualizációjáról szól a \ref{Research}. fejezet.
A \ref{S-graph}. fejezetben részletesebben bemutatásra kerül az egyik megoldó módszer, az S-gráf keretrendszer, amely a dolgozatom további fókuszát képezi.
A szakdolgozat témáját képező sztochasztikus profit maximalizálási probléma a \ref {Problem_def}. fejezetben kerül részletesen definiálásra.
Az \ref{s-graph_framework}. fejezetben az S-gráf keretrendszerben definiált algoritmusok megoldására kifejlesztett S-gráf solver program kerül röviden bemutatásra.
A \ref{Problem_def}. fejezetben definiált probléma megoldásához felhasznált módszerek részletes leírását, valamint az S-gráf solver program továbbfejlesztése közben végzett munka részletes dokumentációját a \ref{Problem_impl}. fejezet tartalmazza.
A solver programon végzett fejlesztési munkák végezte után szükséges tesztelés menetét írja le a \ref{Testing}. fejezet.
A \ref{Summary}. fejezet képezi a dolgozat, valamint az elvégzett munka összefoglalását, értékelését, valamint a jövőbeli esetleges továbbfejlesztési lehetőségek ismertetését.
A dolgozat végén található a melléklet, mely három függelékből áll: az \ref{legend} függelék tartalmazza a jelmagyarázatot, a solver program példa input fájljai a \ref{input_files} függelékben találhatóak, a \ref{cd} függelék tartalmazza a CD melléklet tartalmának leírását.   