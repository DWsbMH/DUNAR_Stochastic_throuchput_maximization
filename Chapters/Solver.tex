\chapter{Az S-gráf solver} \label{s-graph_framework}
Az S-gráf solver program egy C++ nyelven íródott, több szálas megoldó program, mely az S-gráf keretrendszerben foglalt algoritmusok segítségével különböző ütemezési problémákat képes megoldani.
Jelenleg a NIS, UIS, UW, és LW tárolási irányelveket támogatja, valamint a következő célfüggvényekkel használható: makespan minimalizáció, throughput maximalizáció, cycle time minimalizáció.
A solver parancssorból futtatható, különböző parancssori kapcsolók teszik lehetővé a különböző funkciókhoz tartozó paraméterek beállítását.
Néhány, a throughput maximalizáláshoz fontos kapcsoló:
\begin{description}
\item[-i, --input [file]] Az input fájl elérési útja, a fájl kiterjesztése .xml, vagy .ods lehet
\item[-o, --output [file]] Az output fájl elérési útja, a fájl kiterjesztése lehet .txt, vagy .png attól függően, hogy szöveges, vagy grafikus megjelenítést szeretnénk eredményül kapni
\item[--obj [objective function]] A célfüggvény típusa, throughput maximalizálás esetén ezen kapcsoló értéke: thmax
\item[--timehor [time horizon]] A throughput maximalizáláshoz rendelkezésre álló időhorizont mérete 
\end{description}
A fentiek alapján a solver futtatására egy példa:
\begin{figure}[H]
\begin{center}
\includegraphics[scale=0.5]{switchesExample}
\caption{Példa a solver futtatására}
\label{switchExample}
\end{center}
\end{figure}
A solver az OpenMP API-t használja párhuzamosításra, a kód lefordításához Qt5 szükséges.
Ezenfelül Linux alatt történő futtatáshoz letöltendő még a Google OR-Tools könyvtár, Windows operációs rendszer esetén pedig a Visual C++ Redistributable, valamint a boost könyvtár telepítése a követelmény.
A fentiek telepítése, illetve a megfelelő könyvtárak elérési útjainak beállítása után a solver qmake segítségével fordítható, melynek következtében létrejönnek a futtatáshoz szükséges fájlok, köztük a solver.exe (Windows esetén), mellyel immáron futtathatjuk a megoldó programot az \ref{switchExample}. ábrához hasonló módon.   
\section{A solver felépítése}
\section{A determinisztikus throughput maximalizáló} \label{throughput_solver}