\chapter{Teszteredmények}
Ebben a fejezetben a tesztelése menete, a tesztelés által elért eredmények kerülnek bemutatásra.
Mivel esetemben egy meglévő szoftver rendszer továbbfejlesztéséről beszélhetünk, ezért az új funkciók letesztelésén kívül fontos az eddigi funkcionalitás újratesztelése is.
Éppen ezért a tesztesetek alapvetően három részre bonthatóak:
\begin{itemize}
\item A determinisztikus throughput maximalizáló tesztelése
\item A sztochasztikus alapesetek (1-1) tesztelése
\item A sztochasztikus multiproduct esetek tesztelése
\end{itemize}
A teszteléshez használt input fájlok megtalálhatóak a CD melléklet \fileName{Tesztelés/Tesztfájlok} mappájában.
Ezen inputfájlok tartalma, nevezetesen a termékekre, illetve a forgatókönyvekre vonatkozó paraméterek random szám generátorral készültek adott tartományokon belül.
A tesztkonfigurácó leírása:
\begin{itemize}
\item Windows 7 operációs rendszer 
\item Qt 5.11.2, Microsoft Visual C++ Compiler 15.0, Boost Libraries 1.68.0
\item Intel i5 3570K 3,8 Ghz processzor
\item 8 GB RAM
\end{itemize}
\section{A determinisztikus throughput maximalizáló tesztelése}
\section{A sztochasztikus alapesetek tesztelése (1-1)}
\section{A sztochasztikus multiproduct receptek tesztelése}